%%%%%%%%%%%%%%%%%%%%%%%%%%%%%%%%%%%%%%%%%%%%%%%%%%%%%%%%%%%%%%%%%%%%%%%%%
%%
%W  libform.tex              GAP documentation           Alexander Hulpke
%%
%H  @(#)$Id$
%%

%%%%%%%%%%%%%%%%%%%%%%%%%%%%%%%%%%%%%%%%%%%%%%%%%%%%%%%%%%%%%%%%%%%%%%%%%
\Chapter{Library Files}

This chapter describes some of the conventions used in the {\GAP}
library files.
These conventions are intended as a help on how to read library files and
how to find information in them.
So everybody is recommended to obey these conventions,
although they do not prescribe a compulsory programming style
--{\GAP} itself will not bother with the formatting of files.

All filenames used by {\GAP} adhere to the 8+3 convention (this permits to
use the same filenames even on a MS-DOS file system) and are in lower
case (systems that do not recognize lower case in file names will convert
them automatically to upper case).

%%%%%%%%%%%%%%%%%%%%%%%%%%%%%%%%%%%%%%%%%%%%%%%%%%%%%%%%%%%%%%%%%%%%%%%%%
\Section{File Types}

The {\GAP} library consists of the following types of files, distinguished
by their suffixes:

\beginitems
`.g': &
    Files which contain parts of the ``inner workings'' of {\GAP}.
    These files usually do not contain mathematical functionality,
    except for providing links to kernel functions.

`.gd': &
    Declaration files.
    These files contain declarations of all categories, attributes,
    operations, and global functions.
    These files also contain the operation definitions in comments.

`.gi': &
    Implementation files.
    These files contain all installations of methods and global functions.
    Usually declarations of representations are also considered to be
    part of the implementation and are therefore found in the `gi' files.

    &
    As a rule of thumb, all `gd' files are read in before the `gi' files
    are read.
    Therefore a `gi' file usually may use any operation or global function
    (it has been declared before),
    and no care has to be taken towards the order in which the `gi' files
    are read.

`.co': &
    Completion files.
    They are used only to speed up loading
    (see~"ref:Completion Files" in the Reference Manual).
\enditems

%%%%%%%%%%%%%%%%%%%%%%%%%%%%%%%%%%%%%%%%%%%%%%%%%%%%%%%%%%%%%%%%%%%%%%%%%
\Section{File Structure}

Every file starts with a header that lists the filename, copyright, a short
description of the file contents and the original authors of this file.

This is followed by a revision entry:
\begintt
Revision.file_suf :=
    "@(#)$Id$";
\endtt
where `file.suf' is the file name. The revision control system used for the
development will automatically append text to the string ``{`Id: '}'' which
indicates the version number. The reason for these revision entries is to
give the possibility to check from withing {\GAP} for revision numbers of a
file. (Do not mistake these revision numbers for the version number of
{\GAP} itself.)

Global comments usually are indented by two hash marks and two blanks.
If a section of such a comment is introduced by a line containing
a hash mark and a number it will be used for the manual
(stripped of the hash marks and leading two blanks).

Every declaration or method or function installation which is not only of
local scope is introduced by a function header of the following type.
\begintt
#############################################################################
##
#X  ExampleFunction(<A>,<B>)
##
##  This function does nothing.
\endtt
The initial letter (`X' in the example) has the same meaning as at the end
of a declaration line in the Reference Manual (see~"tut:Manual Conventions"
in the Tutorial),
it indicates whether the object declared will be a category, representation
and so forth.
Additionally `M' is used in `.gi' files for method installations.
The line then gives a sample usage of the function.
This is followed by a comment which describes the identifier.
This decription will automatically be used by the process to build the
Reference Manual source.

Indentation in functions and the use of decorative spaces in the code are
left to the decision of the authors of each file .

The file ends with an 
\begintt
#E
\endtt
comment section that may be used to store formatting descriptions for an
editor.

%%%%%%%%%%%%%%%%%%%%%%%%%%%%%%%%%%%%%%%%%%%%%%%%%%%%%%%%%%%%%%%%%%%%%%%%%
%%
%E

